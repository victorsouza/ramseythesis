%%%%%%%%%%%%%%%%%%%%%%%%%%%%%%%%%%%%%%%%%%%%%%%%%%%%%%%%%%%%%%%%%%%%%%%%%%%%%%%%
%% Projeto Final de Graduação
%% Aluno: Victor Seixas Souza
%% Orientadora: Christiane Neme Campos
%% Tema: Teoria de Ramsey em Grafos
%%%%%%%%%%%%%%%%%%%%%%%%%%%%%%%%%%%%%%%%%%%%%%%%%%%%%%%%%%%%%%%%%%%%%%%%%%%%%%%%
% !TEX root = thesis.tex
%%%%%%%%%%%%%%%%%%%%%%%%%%%%%%%%%%%%%%%%%%%%%%%%%%%%%%%%%%%%%%%%%%%%%%%%%%%%%%%%

\chapter{Prefácio}

A Teoria de Ramsey é uma área da matemática que unifica o tema: desordem completa é impossível. Mais especificamente, observamos que se uma estrutura é grande o suficiente, então ela possui uma subestrutura bem especial e ordenada. Este fenômeno ocorre em diversos campos da Matemática, como Combinatória, Geometria e Teoria dos Números. Este projeto aborda conceitos básicos e alguns dos resultados clássicos em Teoria de Ramsey aplicada a grafos.

O exemplo mais simples de tal corpo teórico é frequentemente apresentado na seguinte história: em uma festa com pelo menos seis pessoas, três delas se conhecem mutuamente ou três delas não se conhecem mutuamente. Se enxergarmos a relação de conhecer como simétrica, o mesmo pode ser traduzido para linguagem de Teoria de Grafos como: todo grafo com pelo menos seis vértices possui um triângulo, ou, então, o seu grafo complementar possui um triângulo. A Teoria de Ramsey inicia-se pela generalização sucessiva deste enunciado para grafos e hipergrafos.

A Teoria de Ramsey tem seu nome em homenagem ao matemático e filósofo britânico Frank P. Ramsey, por seu trabalho, em lógica, publicado em 1930 \cite{ramsey}, mas apenas adquiriu um corpo teórico coeso na década de 1970. A área vem recebendo grande atenção nos últimos vinte anos por suas conexões com diversos campos da matemática e, ainda assim, muito dos seus problemas fundamentais permanecem sem solução. Além disso, não muito da teoria propagou-se para os livros didáticos em nível de graduação. Entretanto, é possível abordar parte da teoria sem recorrer ao ferramental mais avançado.

Considerando a lacuna da literatura citada anteriormente, este texto tem por objetivo apresentar conceitos básicos da Teoria de Ramsey em grafos em um nível introdutório e em língua portuguesa. Além disso, planeja-se completar este texto com a apresentação de um ou dois tópicos mais avançados da área, que serão selecionados dentre: o método probabilístico; o lema da regularidade de Szemerédi; ou aplicações em Teoria dos Números.

%%%%%%%%%%%%%%%%%%%%%%%%%%%%%%%%%%%%%%%%%%%%%%%%%%%%%%%%%%%%%%%%%%%%%%%%%%%%%%%%
