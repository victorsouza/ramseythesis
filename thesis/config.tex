%%%%%%%%%%%%%%%%%%%%%%%%%%%%%%%%%%%%%%%%%%%%%%%%%%%%%%%%%%%%%%%%%%%%%%%%%%%%%%%%
%% Projeto Final de Graduação
%% Aluno: Victor Seixas Souza
%% Orientadora: Christiane Neme Campos
%% Tema: Teoria de Ramsey em Grafos
%%%%%%%%%%%%%%%%%%%%%%%%%%%%%%%%%%%%%%%%%%%%%%%%%%%%%%%%%%%%%%%%%%%%%%%%%%%%%%%%
% !TEX root = thesis.tex
%%%%%%%%%%%%%%%%%%%%%%%%%%%%%%%%%%%%%%%%%%%%%%%%%%%%%%%%%%%%%%%%%%%%%%%%%%%%%%%%
%% PACKAGES
\usepackage{techrep-PFG-ic}
%% FIGURES
\usepackage[usenames,dvipsnames]{xcolor}
\usepackage{tkz-graph}
\usepackage{tkz-berge}
%% LANGUAGE
\usepackage[brazilian]{babel}
\usepackage[utf8]{inputenc}
\usepackage[T1]{fontenc}
%% PAGE
%\usepackage{fullpage}
%% MATH
\usepackage{amsmath}
\usepackage{amssymb}
\usepackage{amsfonts}
\usepackage{amsthm}
\usepackage{nicefrac}
\usepackage{bbm}
\usepackage{mathtools}
%% NOMENCLATURE
\usepackage{nomencl}
%% CHAPTER HEADINGS
\usepackage[Sonny]{fncychap}
%% TODONOTES
\usepackage{todonotes}
%% TABLE AND FIGURE COUNTERS
\usepackage{chngcntr}
%% HEADINGS
\usepackage{fancyhdr}

\usepackage{enumitem}
\usepackage{float}

%%%%%%%%%%%%%%%%%%%%%%%%%%%%%%%%%%%%%%%%%%%%%%%%%%%%%%%%%%%%%%%%%%%%%%%%%%%%%%%%
%% FILE INFO

\newcommand*\samethanks[1][\value{footnote}]{\footnotemark[#1]}

\author{Autor: Victor Seixas Souza \thanks{Instituto  de Computação, Universidade Estadual  de Campinas, Campinas}\\
Supervisora: Christiane Neme Campos \samethanks}

\title{\Huge{Introdução à Teoria de Ramsey em Grafos}}

\date{\today}

%%%%%%%%%%%%%%%%%%%%%%%%%%%%%%%%%%%%%%%%%%%%%%%%%%%%%%%%%%%%%%%%%%%%%%%%%%%%%%%%
%% CONFIGURATION

\bibliographystyle{acm}
%\makenomenclature
\pagestyle{fancy}
\fancyhf{}
\fancyhead[LE,RO]{\thepage}
\fancyhead[CE]{\leftmark}
\fancyhead[CO]{\rightmark}
\renewcommand{\headrulewidth}{1pt}

\makeatletter
\renewcommand\mainmatter{\clearpage\@mainmattertrue\pagenumbering{arabic}}
\makeatother

%%%%%%%%%%%%%%%%%%%%%%%%%%%%%%%%%%%%%%%%%%%%%%%%%%%%%%%%%%%%%%%%%%%%%%%%%%%%%%%%
%% THEOREMS

\newtheorem{theorem}{Teorema}%[chapter]
\newtheorem{definition}[theorem]{Definição}
\newtheorem{lemma}[theorem]{Lema}
\newtheorem{proposition}[theorem]{Proposição}
\newtheorem{corollary}[theorem]{Corolário}
\newtheorem{fact}[theorem]{Fato}
\newtheorem{observation}[theorem]{Observação}
\newtheorem{openproblem}[theorem]{Problema em Aberto}
\newtheorem{example}[theorem]{Exemplo}

%%%%%%%%%%%%%%%%%%%%%%%%%%%%%%%%%%%%%%%%%%%%%%%%%%%%%%%%%%%%%%%%%%%%%%%%%%%%%%%%
%% MATH OPERATORS

\DeclareMathOperator{\prob}{\mathbf{P}}
\DeclareMathOperator{\expec}{\mathbf{E}}
\DeclareMathOperator{\var}{Var}
\DeclareMathOperator{\cover}{cov}
\DeclareMathOperator{\hit}{hit}
\DeclareMathOperator{\taucover}{\tau_{\cover}}
\DeclareMathOperator{\tcover}{t_{\cover}}
\DeclareMathOperator{\tauhit}{\tau_{\hit}}
\DeclareMathOperator{\thit}{t_{\hit}}
\DeclareMathOperator{\mdc}{mdc}
\DeclareMathOperator{\binomial}{Binom}
\DeclareMathOperator{\poisson}{Poisson}
\DeclareMathOperator{\exponential}{Exp}
\DeclareMathOperator{\Borel}{\mathcal{B}}
\DeclareMathOperator{\R}{\mathbb{R}}
\DeclareMathOperator{\R+}{\mathbb{R}_+}

\DeclareMathOperator{\union}{\cup}
\DeclareMathOperator{\iso}{\simeq}
\DeclareMathOperator{\indep}{\perp}
\DeclareMathOperator{\lequiv}{\Longleftrightarrow}
\newcommand{\comp}[1]{\overline{#1}}
\newcommand{\card}[1]{\left | #1 \right |}
\newcommand{\Mod}[1]{\ (\text{mod}\ #1)}
\newcommand{\legendre}[2]{\big(\frac{#1}{#2}\big)}
\newcommand{\Z}[1]{\mathbb{Z}_{#1}}
\newcommand{\ind}[1]{\mathbbm{1}[#1]}

%%%%%%%%%%%%%%%%%%%%%%%%%%%%%%%%%%%%%%%%%%%%%%%%%%%%%%%%%%%%%%%%%%%%%%%%%%%%%%%%
%% BOUND TABLE
%\newcommand{\tbound}[2]{\begin{tabular}[c]{@{}l@{}} $\geq#1$\\ $\leq #2$\end{tabular}}
%\newcommand{\texact}[1]{$=#1$}
\newcommand{\tbound}[2]{\begin{tabular}[c]{@{}l@{}} {\color{Red}\textbf{#1}} \\ {\color{Blue}\textbf{#2}}\end{tabular}}
\newcommand{\texact}[1]{{\color{ForestGreen}\textbf{#1}}}

\counterwithout{figure}{chapter}
\counterwithout{table}{chapter}


%%%%%%%%%%%%%%%%%%%%%%%%%%%%%%%%%%%%%%%%%%%%%%%%%%%%%%%%%%%%%%%%%%%%%%%%%%%%%%%%
%% INILINE DEFINITIONS
\newcommand{\indef}[1]{\emph{#1}}
