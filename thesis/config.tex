%%%%%%%%%%%%%%%%%%%%%%%%%%%%%%%%%%%%%%%%%%%%%%%%%%%%%%%%%%%%%%%%%%%%%%%%%%%%%%%%
%% Projeto Final de Graduação
%% Aluno: Victor Seixas Souza
%% Orientadora: Christiane Neme Campos
%% Tema: Teoria de Ramsey em Grafos
%%%%%%%%%%%%%%%%%%%%%%%%%%%%%%%%%%%%%%%%%%%%%%%%%%%%%%%%%%%%%%%%%%%%%%%%%%%%%%%%
%% PACKAGES

%% LANGUAGE
\usepackage[brazilian]{babel}
\usepackage[utf8]{inputenc}
\usepackage[T1]{fontenc}
%% PAGE
%\usepackage{fullpage}
%% MATH
\usepackage{amsmath}
\usepackage{amssymb}
\usepackage{amsfonts}
\usepackage{amsthm}
%% NOMENCLATURE
\usepackage{nomencl}
%% CHAPTER HEADINGS
\usepackage[Sonny]{fncychap}
%% TODONOTES
\usepackage{todonotes}

%%%%%%%%%%%%%%%%%%%%%%%%%%%%%%%%%%%%%%%%%%%%%%%%%%%%%%%%%%%%%%%%%%%%%%%%%%%%%%%%
%% FILE INFO

\author{Aluno: Victor Seixas Souza\\
Supervisora: Christiane Neme Campos}

\title{Projeto Final de Graduação\\
\Huge{Introdução à Teoria de Ramsey em Grafos}}

\date{\today}

%%%%%%%%%%%%%%%%%%%%%%%%%%%%%%%%%%%%%%%%%%%%%%%%%%%%%%%%%%%%%%%%%%%%%%%%%%%%%%%%
%% CONFIGURATION

\pagestyle{plain}
\bibliographystyle{plain}
%\makenomenclature

%%%%%%%%%%%%%%%%%%%%%%%%%%%%%%%%%%%%%%%%%%%%%%%%%%%%%%%%%%%%%%%%%%%%%%%%%%%%%%%%
%% THEOREMS

\newtheorem{definition}{Definição}
\newtheorem{theorem}{Teorema}
\newtheorem{lemma}[theorem]{Lema}
\newtheorem{corollary}[theorem]{Corolário}
\newtheorem{fact}[theorem]{Fato}
\newtheorem{observation}[theorem]{Observação}
\newtheorem{example}{Exemplo}

%%%%%%%%%%%%%%%%%%%%%%%%%%%%%%%%%%%%%%%%%%%%%%%%%%%%%%%%%%%%%%%%%%%%%%%%%%%%%%%%
%% MATH OPERATORS

\DeclareMathOperator{\prob}{\mathbf{P}}
\DeclareMathOperator{\expec}{\mathbf{E}}
\DeclareMathOperator{\var}{Var}
\DeclareMathOperator{\cover}{cov}
\DeclareMathOperator{\hit}{hit}
\DeclareMathOperator{\taucover}{\tau_{\cover}}
\DeclareMathOperator{\tcover}{t_{\cover}}
\DeclareMathOperator{\tauhit}{\tau_{\hit}}
\DeclareMathOperator{\thit}{t_{\hit}}
\DeclareMathOperator{\mdc}{mdc}
\DeclareMathOperator{\binomial}{Binom}
\DeclareMathOperator{\poisson}{Poisson}
\DeclareMathOperator{\exponential}{Exp}
\DeclareMathOperator{\Borel}{\mathcal{B}}
\DeclareMathOperator{\R}{\mathbb{R}}
\DeclareMathOperator{\R+}{\mathbb{R}_+}

\DeclareMathOperator{\union}{\cup}
\newcommand{\comp}[1]{\overline{#1}}
\newcommand{\card}[1]{\left | #1 \right |}
