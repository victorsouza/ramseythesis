%%%%%%%%%%%%%%%%%%%%%%%%%%%%%%%%%%%%%%%%%%%%%%%%%%%%%%%%%%%%%%%%%%%%%%%%%%%%%%%%
%% Projeto Final de Graduação
%% Aluno: Victor Seixas Souza
%% Orientadora: Christiane Neme Campos
%% Tema: Teoria de Ramsey em Grafos
%%%%%%%%%%%%%%%%%%%%%%%%%%%%%%%%%%%%%%%%%%%%%%%%%%%%%%%%%%%%%%%%%%%%%%%%%%%%%%%%
% !TEX root = ../thesis.tex
%%%%%%%%%%%%%%%%%%%%%%%%%%%%%%%%%%%%%%%%%%%%%%%%%%%%%%%%%%%%%%%%%%%%%%%%%%%%%%%%

\chapter{Método Probabilístico}

%%%%%%%%%%%%%%%%%%%%%%%%%%%%%%%%%%%%%%%%%%%%%%%%%%%%%%%%%%%%%%%%%%%%%%%%%%%%%%%%

Recall de fatos de probabilidade.

%%%%%%%%%%%%%%%%%%%%%%%%%%%%%%%%%%%%%%%%%%%%%%%%%%%%%%%%%%%%%%%%%%%%%%%%%%%%%%%%

\section{Limitantes Inferiores}

Os limitantes inferiores visto na seção anterior anterior são construtivos, isto é, quando mostramos que $R(4,4) > 17$, por exemplo, isto passou pela construção de uma coloração de arestas do grafo $K_{17}$. Para encontrar um bom limitante inferior, vamos precisar de uma abordagem um pouco diferente.

O nosso objetivo é encontrar colorações de arestas de $K_n$ sem $K_k$ monocromáticos com $k$ fixado e $n$ maior possível, para poder concluir que $R(k,k) > n$. Note que não necessariamente precisamos construir tal coloração, basta saber que alguma coloração que satisfaz estas propriedades existe. Vamos introduzir agora uma técnica chamada de \indef{método probabilistico} para resolver este tipo de problema. O nosso objetivo é provar que uma estrutura com certas propriedades existe. Definimos então um espaço de probabilidade sobre o conjunto destas estruturas de maneira adequada e mostramos que com probabilidade positiva, a propriedade de interesse é satisfeita. Isto nos mostra que o conjunto dos elementos cuja propriedade são satisfeitas não pode ser vazio, caso contrário, a probabilidade seria 0, e mostramos a existência de uma estrutura com esta propriedade.

Esta técnica foi aplicada com sucesso pela primeira vez em 1947 pelo matemático Paul Erdös para encontrar justamente um limitante inferior para os números de Ramsey diagonais $R(k,k)$~\cite{erdos47}. Antes do tal, vamos revisar alguns conceitos de probabilidade discreta, em particular, estamos interessados em escolher uma coloração de arestas do $K_n$ de maneira aleatória. Seja $\mathcal{C}$ o conjunto de todas as colorações de arestas em duas cores do $K_n$. Claramente, temos que $|\mathcal{C}| = 2^{\binom{n}{2}}$.
Suponha que escolhemos uma coloração $c \in \mathcal{C}$ aleatoria e uniformemente, queremos saber qual é a probabilidade desta coloração possuir uma dada propriedade. Seja $\mathcal{P} \subset \mathcal{C}$ o conjunto das colorações que satisfazem a dada propriedade. Temos então que $\prob(c \text{ possui a propriedade}) = \prob(c \in \mathcal{P}) = |\mathcal{P}| / |\mathcal{C}|$, onde $\prob$ é a medida de probabilidade.
Em particular, se $e$ é uma aresta, então $|\mathcal{P}| = |\{c \in \mathcal{C} : c(e) = R \}| = 2^{\binom{n}{2}-1}$, o que nos dá $\prob(c(e) = R) = \frac{1}{2}$ e também $\prob(c(e) = B) = \frac{1}{2}$. Podemos agora introduzir a primeira aplicação do método probabilístico:

%%%%%%%%%%%%%%%%%%%%%%%%%%%%%%%%%%%%%%%%
\begin{theorem}[Erdös, 1947]
Se $\displaystyle \binom{n}{k} 2^{1 - \binom{k}{2}} < 1$ então $R(k,k) > n$. Isso nos dá que $R(k,k) > \lfloor 2^{k/2} \rfloor$ para todo $k \geq 3$.
\end{theorem}
%%%%%%%%%%%%%%%%%%%%%%%%%%%%%%%%%%%%%%%%
\begin{proof}
Considere uma coloração de arestas do $K_n$ em duas cores obtida de maneira aleatória colorindo independentemente e uniformemente. Fixe um subconjunto $A \subset V(K_n)$ de $k$ vértices e seja $M_A$ o evento de que o subrafo completo induzido por $A$ é monocromático. Fixando uma cor para as arestas do subgrafo induzido por $A$, podemos colorir as diversas $\binom{n}{2} - \binom{k}{2}$ arestas de qualquer cor. Como existem duas cores disponíveis, temos:
\[ \prob(M_A) = \frac{2\cdot 2^{\binom{n}{2} - \binom{k}{2}}}{2^{\binom{n}{2}}}  = 2^{1 - \binom{k}{2}} \]
Considere agora o evento $M = \bigcup_{R \subset V(G), |R|=k}M_R$ que corresponde a existência de algum $K_k$ monocromático.
\begin{align*}
\prob(M) &= \prob\Bigg( \;\;\;\;\bigcup_{\mathclap{\substack{R \subset V(K_n)\\ |R| = k}}}M_R \Bigg) \leq \sum_{\mathclap{\substack{R \subset V(K_n)\\ |R| = k}}}\prob(M_R) \\
&= \sum_{\mathclap{\substack{R \subset V(K_n)\\ |R| = k}}} 2^{1 - \binom{k}{2}} = \binom{n}{k} 2^{1 - \binom{k}{2}} < 1
\end{align*}
Logo, $\prob(\comp{M_R}) = 1 - \prob(M_R) > 0$, o que nos indica que existe alguma coloração de $K_n$ sem $K_k$ monocromático. Com isso, temos $R(k,k) > n$.

Lembre que $\binom{n}{k} \leq n^k /k!$ e suponha que $n = \lfloor 2^{k/2} \rfloor$. Temos então que
\[ \binom{n}{k}2^{1 - \binom{k}{2}} \leq \frac{n^k}{k!}2^{1 - \frac{k^2}{2} + \frac{k}{2}} = \frac{n^k 2^{1 + \frac{k}{2}}}{k!2^{\frac{k^2}{2}}} \leq \frac{2^{1 + \frac{k}{2}}}{k!} < 1 \]
quando $n \geq 3$. Assim, temos $R(k,k) > \lfloor 2^{k/2} \rfloor$ quando $n \geq 3$.
\end{proof}
%%%%%%%%%%%%%%%%%%%%%%%%%%%%%%%%%%%%%%%%

Este tipo de idéia pode ser refinada para obter resultados melhores para o limitante inferior. No entanto, não se conhece algum limitante inferior cuja ordem exponencial seja superior à $\sqrt{2}^k$. Vimos na seção anterior, que o melhor limitante superior conhecido possui ordem exponencial $4^k$. Um dos problemas abertos mais importantes em Combinatória é determinar a ordem exponencial correta para o crescimento dos números de Ramsey diagonais.

Na seção anterior, vimos alguns limitantes superiores para o número $R(k,k)$ que tem ordem assintótica próxima de $4^k$.

%%%%%%%%%%%%%%%%%%%%%%%%%%%%%%%%%%%%%%%%
\begin{openproblem}
Constructive Ramsey
\end{openproblem}
%%%%%%%%%%%%%%%%%%%%%%%%%%%%%%%%%%%%%%%%

%%%%%%%%%%%%%%%%%%%%%%%%%%%%%%%%%%%%%%%%%%%%%%%%%%%%%%%%%%%%%%%%%%%%%%%%%%%%%%%%

\section{O Lema Local de Lovász}

%%%%%%%%%%%%%%%%%%%%%%%%%%%%%%%%%%%%%%%%%%%%%%%%%%%%%%%%%%%%%%%%%%%%%%%%%%%%%%%%
