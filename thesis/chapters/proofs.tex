%%%%%%%%%%%%%%%%%%%%%%%%%%%%%%%%%%%%%%%%%%%%%%%%%%%%%%%%%%%%%%%%%%%%%%%%%%%%%%%%
%% Projeto Final de Graduação
%% Aluno: Victor Seixas Souza
%% Orientadora: Christiane Neme Campos
%% Tema: Teoria de Ramsey em Grafos
%%%%%%%%%%%%%%%%%%%%%%%%%%%%%%%%%%%%%%%%%%%%%%%%%%%%%%%%%%%%%%%%%%%%%%%%%%%%%%%%

\chapter{Provas}

Neste capítulo, estão temporariamente os resultados que foram julgados interessantes e suas provas preliminares.\cite{alon}

%%%%%%%%%%%%%%%%%%%%%%%%%%%%%%%%%%%%%%%%%%%%%%%%%%%%%%%%%%%%%%%%%%%%%%%%%%%%%%%%

\section{Números Pequenos}

Denotaremos por $K_n$ o grafo completo com $n$ vértices. Uma coloração das arestas de $K_n = (V,E)$ em duas cores é um mapa $c: E \to \{R,B\}$ onde $R$ e $B$ indicam as cores vermelho e azul respectivamente. Uma coloração de arestas induz uma decomposição dada por $K_n = E_R \union E_B$

\begin{definition}
Seja $G$ um grafo
\end{definition}

%%%%%%%%%%%%%%%%%%%%%%%%%%%%%%%%%%%%%%%%%%%%%%%%%%%%%%%%%%%%%%%%%%%%%%%%%%%%%%%%
