%%%%%%%%%%%%%%%%%%%%%%%%%%%%%%%%%%%%%%%%%%%%%%%%%%%%%%%%%%%%%%%%%%%%%%%%%%%%%%%%
%% Projeto Final de Graduação
%% Aluno: Victor Seixas Souza
%% Orientadora: Christiane Neme Campos
%% Tema: Teoria de Ramsey em Grafos
%%%%%%%%%%%%%%%%%%%%%%%%%%%%%%%%%%%%%%%%%%%%%%%%%%%%%%%%%%%%%%%%%%%%%%%%%%%%%%%%
% !TEX root = ../thesis.tex
%%%%%%%%%%%%%%%%%%%%%%%%%%%%%%%%%%%%%%%%%%%%%%%%%%%%%%%%%%%%%%%%%%%%%%%%%%%%%%%%

\chapter{Introdução}

No primeiro dia de aula em uma escolinha de matemática, a professora preparou uma gincana para entrosar os alunos. Ela os sorteou em grupos de seis, e cada grupo deveria se sentar em mesas separadas. O objetivo era que os alunos conversasem entre sí sobre o que fizeram durante as férias e potencialmente formassem novas amizades. Alguns alunos haviam sido colegas nos anos anteriores, e portanto já se conheciam. Outros alunos vieram de turmas diferentes, então não conheciam à todos. Além disso, novos alunos entram na escolinha todo ano para aprender matemática!

Após o sorteio, a professora supervisionou atentamente os grupos, e observou que em algumas mesas, três alunos já se conheciam. Em algumas mesas, até quatro alunos já se conheciam. Preocupada se a gincana teria o efeito desejado, ela adotou outra estratégia e focou apenas nos alunos que ainda não se conhecem. Em algumas mesas, haviam três alunos que ainda não se conheciam, mas em outras pareciam menos promissoras, não havendo nem três alunos que não se conhecem. Curioso que estas mesas haviam três alunos que se conheciam. Sendo uma professora de matemática, ela se perguntou se existe algum motivo para isto acontecer. Será que se ela fizesse outro sorteio, poderia haver uma mesa em que nenhum grupo de três alunos se conheçem e nenhum grupo de três alunos não se conheçam? Por sorte, a professora havia estudado Teoria de Grafos, e conseguiu verificar o seguinte fato:

% Suponha que em uma festa, seis pessoas se sentam em uma mesa para conversar. Dentre estas pessoas, algumas já se conhecem, enquanto outras estão se encontrando pela primeira vez, nesse caso dizemos que elas são estranhas. A priori, não há muito o que se possa concluir sobre estas pessoas: todas podem se conhecer, todas podem ser estranhas entre si, ou algumas se conheçam e outras não. De qualquer forma, podemos encontrar algo de interessante na mesa se procurarmos pela estrutura correta, como a seguinte observação que faz parte do folclore matemático:

\begin{fact}
Em uma mesa com seis alunos, três alunos se conhecem ou três alunos não se conhecem.
\end{fact}

Vamos agora como que a professora percebeu este fato. Considere que você seja um dos seis alunos da mesa. Dos cinco restantes, você conhece uma quantidade, digamos $C$, e também não conhece uma quantidade $N$. Sabemos que $C + N = 5$. Isto nos dá que ou $C \geq 3$ ou $N \geq 3$. De fato, se $C \geq 2$ e $N \geq 2$ occorrerem simultaneamente, $5 = C + N \geq 4$, o que é absurdo. Suponha que $C \geq 3$, isto significa que você conhece pelo menos três dos outros alunos, digamos Alberto, Bruna e Carlos Se existe alguma amizade entre eles, digamos, Alberto e Carlos são amigos, então a mesa possui um grupo de três alunos que se conhece, você, Alberto e Carlos. Caso isto não ocorra, então Alberto, Bruna e Carlos formam um grupo de seis alunos que não se conhecem. Analogamente, temos o mesmo raciocínio quando $N \geq 3$.

Este fato pode ser considerado o primeiro resultado na Teoria de Ramsey e possui generalizações interessantes. Além disso, ele possui uma formulação mais clara em termos de coloração de grafos. Na próxima seção, introduziremos os conceitos da Teoria de Grafos que serão necessários para adentrar o universo da Teoria de Ramsey em grafos. Um leitor familiarizado com Grafos é convidado a fazer uma leitura rápida da seção para atentar à notação utilizada neste texto.

%\todo{reescrever paragrafo} O leitor está convidado à tentar demonstrar este fato por sí só. Vamos ver em breve que este fato pode ser modelado naturalmente com colorações de arestas de um grafos. Assim, não vamos somente provar este fato, mas também diversas generalizações dele. Para o tal, precisamos introduzir alguns conceitos e notações em Teoria de Grafos que o leitor experiente pode pular.


%%%%%%%%%%%%%%%%%%%%%%%%%%%%%%%%%%%%%%%%%%%%%%%%%%%%%%%%%%%%%%%%%%%%%%%%%%%%%%%%

\section{Noções de Grafos}

Um \indef{grafo} é um par ordenado $G = (V(G), E(G))$, que consiste de um conjunto $V(G)$ de vértices, e um conjunto $E(G)$ de arestas, disjunto de $V(G)$, em conjunto com uma função de incidência $\Psi_G$ que associa as arestas de $G$ à pares de vértices $\{a,b\}$ não necessariamente distintos.

Uma aresta $e \in E(G)$ e vértices $u, v \in V(G)$ são ditos \indef{incidentes} se $\Psi_G(e) = \{u,v\}$. Neste caso, diz-se ainda que $u$ e $v$ são as \indef{extremidades} de $e$. A incidência relaciona elementos de conjuntos distintos, neste caso, vértices e arestas.
Uma outra relação é a de adjacência, que se aplica para elementos de mesma natureza. Dois vértices $u$ e $v$ são ditos \indef{adjacentes} em $G$ se existe uma aresta $e$ em $G$ tal que $\Psi_G(e) = \{u,v\}$. Duas arestas $e$ e $w$ são ditas \indef{adjacentes} em $G$ se existe um vértice $v$ em $G$ tal que $v = \Psi_G(e) \cap \Psi_G(w)$, isto é, se existe um vértice $v$ que incide simultaneamente à $e$ e $w$.

Vamos construir um grafo $G = (V(G), E(G))$ para clarificar as definições. O conjunto de vértices será $V(G) = \{u,v,w,x\}$ e o conjunto de arestas será $E(G) = \{e, f, g, h, j, k, l, s\}$. Falta definir a função de incidência $\Psi_G$, que associa as arestas aos vértices:
\begin{align*}
\Psi_G(e) = \{v,x\} & &
\Psi_G(f) = \{u,v\} & &
\Psi_G(g) = \{u,v\} & &
\Psi_G(h) = \{v,w\} \\
\Psi_G(j) = \{v,w\} & &
\Psi_G(k) = \{u,x\} & &
\Psi_G(l) = \{w,x\} & &
\Psi_G(s) = \{x\}
\end{align*}

Isto completa a definição do grafo $G$. Uma maneira interessante de representar grafos é através de um desenho. A Figura \ref{fig:grafo} representa o grafo $G$ da seguinte maneira. Os vértices são indicados por círculos e as arestas são indicadas por linhas que unem os círculos que correspondem aos vértices que a aresta incide. A representação gráfica é muito importante pois nos permite criar uma intuição sobre os conceitos sobre grafos. Vale a pena notar que não existe um único desenho para um grafo: a posição dos vértices e das arestas não é importante.

\begin{figure}[h!]
\centering
\begin{tikzpicture}
\SetGraphUnit{2}
\GraphInit[vstyle=Normal]
\tikzset{LabelStyle/.style={opacity=0,text opacity=1}}
\Vertex[L=$u$]{A}
\EA[L=$v$](A){B}
\EA[L=$w$](B){C}
\SetGraphUnit{3}
\NO[L=$x$](B){D}
\Edge[label=$e$,labelstyle=left](B)(D)
\tikzset{EdgeStyle/.append style = {bend left}}
\Edge[label=$f$,labelstyle=above](A)(B)
\Edge[label=$g$,labelstyle=above](B)(A)
\Edge[label=$h$,labelstyle=above](B)(C)
\Edge[label=$j$,labelstyle=above](C)(B)
\Edge[label=$k$,labelstyle=left](A)(D)
\Edge[label=$l$,labelstyle=right](D)(C)
\Loop[dist=1cm,dir=EA,style=-,label=$s$,labelstyle=above](D)
\end{tikzpicture}
\caption{Representação gráfica do grafo $G$}
\label{fig:grafo}
\end{figure}

O desenho de um grafo pode ser um bom representativo da estrutura do grafo, mas até mesmo grafos que tem o mesmo desenho podem ser diferentes. Dois grafos $G$ e $H$ são idênticos quando $G = H$ no sentido da Teoria de Conjuntos, isto é, $V(G)$ e $V(H)$ são o mesmo conjunto, $E(G)$ e $E(H)$ são o mesmo conjunto e $\Psi_G$ e $\Psi_H$ são a mesma função. Essa noção de igualdade é restritiva demais pois até grafos que possuem mesmo desenhos podem ser diferentes se os rótulos utilizados forem diferentes. Por exemplo, os grafos $G$ e $H$ da Figura \ref{fig:iso} não são identicos pois $V(G) \neq V(H)$, no entanto, eles possuem efetivamente a mesma estrutura.

Para capturar a noção de que dois grafos diferentes podem ter a mesma estrutura, definimos um \indef{isomorfismo} entre dois grafos $G$ e $H$ como um par de bijeções $\theta : V(G) \to V(H)$ e $\phi : E(G) \to E(G)$ tais que elas presvem a relação de incidência, isto é, $\Psi_G(e) = \{u,v\}$ implica que $\Psi_H(\phi(e)) = \{\theta(u),\theta(v)\}$. Quando existe um isomorfismo entre dois grafos $G$ e $H$, dizemos que eles são \indef{isomorficos}, e escrevemos $G \iso H$.

\begin{figure}[h!]
\centering
\begin{tikzpicture}
\GraphInit[vstyle=Normal]
\SetVertexMath
\begin{scope}[rotate=90]
  \grComplete[prefix=a,RA=2.3]{5}
\end{scope}
\begin{scope}[xshift=6cm, rotate=90]
  \grComplete[prefix=b,RA=2.3]{5}
\end{scope}
\draw (0,-2.5) node [below]{$G$};
\draw (6,-2.5) node [below]{$H$};
\end{tikzpicture}
\caption{Dois grafos não identicos $G$ e $H$ que são isomorficos}
\label{fig:iso}
\end{figure}

Grafos isomorficos possuem essencialmente a mesma estrutura e são considerados como iguais para todos os efeitos práticos. Com efeito, muitas vezes, os nomes que os vértices possuem não tem nenhuma importância. Quando abrimos mão de definir os nomes dos vértices, temos um \indef{grafo não rotulado}, em contrapartida com os \indef{grafos rotulados}. Muitas vezes, apenas nos interessa a interconectividade dos vértices, então a maioria das vezes consideraremos grafos não rotulados e daremos nomes aos vértices conforme necessário.

Formalmente, a relação de isomorfismo $\iso$ é uma relação de equivalência e as classes de equivalência são os grafos não rotulados. A medida que necessitamos por rótulos em alguns vértices, estamos escolhendo algum representante dentro desta classe de equivalência que seja mais conveniente para nossos propósitos.

% TOO MUCH BREAK

Até agora, a definição de grafos que temos é bastante abrangente e permite algumas situações que não gostariamos de considerar em nosso estudo. A primeira situação especial ocorre quando duas arestas $e$ e $w$ incidem nos mesmos vértices, isto é, $\Psi_G(e) = \Psi_G(w)$, e dizemos que essas arestas são \indef{paralelas}. No grafo exemplificado na Figura \ref{fig:grafo}, $f$ e $g$ são arestas paralelas. A segunda situação é quando $e$ incide em apenas um vértice $v$, isto é, $\Psi_G(e) = \{v\}$, e dizemos que $e$ é um \indef{laço}, e no grafo da Figura \ref{fig:grafo}, $s$ é um laço. Um grafo no qual não existem arestas paralelas ou laços é dito um \indef{grafo simples}, e ele será o objeto de nosso estudo.

Seja $G$ um grafo simples e $u$ e $v$ dois vértices de $G$. Se $u$ e $v$ forem adjacentes, então existe uma única aresta $e$ em $G$ com $\Psi_G(e) = \{u,v\}$, portanto podemos identificar $e$ por $\{u,v\}$. De fato, quando $G$ é simples, a função de incidência $\Psi_G$ é redundante. Portanto, ao definir grafos simples, pode-se omitir a função de incidência $\Psi$ definindo o conjunto das arestas $E(G)$ como um subconjunto de pares de vertices. Para tal propósito, para um conjunto $S$, definimos o conjunto de pares de elementos de $S$ por $S^{(2)}$, assim, podemos ver um grafo como um par $G = (V(G),E(G))$ onde $E(G) \subset V(G)^{(2)}$.

Para simplificar a notação, quando não houver ambiguidade, podemos nos referir ao conjunto de vértices apenas por $V$ e o conjunto de arestas por $E$ em um grafo $G = (V,E)$. A quantidade de vértices de um garfo é chamada de \indef{ordem} do grafo, costuma ser denotada por $n$ ou $v(G)$. Já a quantidade de arestas em um grafo é chamada de \indef{tamanho} do grafo, e é denotado por $m$ ou $e(G)$. Se $G$ é um grafo simples, observamos que $E \subset V^{(2)}$ implica que $m \leq \binom{n}{2} = O(n^2)$, isto é, um grafo simples tem uma quantidade de arestas limitada ao quadrado do número de vértices em ordem.

Se $v$ é um vértice de um grafo $G$, então a vizinhança $N_G(v)$ de $v$ é o conjunto dos vértices adjacentes à $v$. O grau de um vértice $d_G(v)$ é o tamanho de sua vizinhança, isto é, $d_G(v) = \card{N_G(v)}$. Quando o grafo for conhecido pelo contexto, utilizaremos apenas $N(v)$ e $d(G)$.

\begin{enumerate}
\item Famílias de grafos
\item Grafos completos
\item Complementar de grafos
\item Coloração de arestas
\end{enumerate}

%Seja $V$ um conjunto finito, denotamos por $V^{(k)}$ o conjunto dos subconjuntos de $V$ com $k$ elementos. Por exemplo, $V^{(2)}$ é o conjunto de pares de elementos de $V$, isto é, elementos de $V^{(2)}$ tem a forma $\{ x,y\}$ onde $x,y\in V$ são elementos distintos. Note que se $V$ possui $n$ elementos, então $V^{(k)}$ possui $\binom{n}{k}$ elementos.

%Um grafo é um par $G = (V,E)$ onde $V$ é um conjunto finito e não vazio de vértices e $E$ é um subconjunto de $V^{(2)}$, cujos elementos são chamados de arestas. Na literatura de Teoria de Grafos, esta definição corresponde à definição de grafos simples e finitos.

%Uma aresta $\{ x,y\}$ poderá ser denotada por $xy$ ou $yx$. Dizemos que a aresta $xy$ incide nos vértices $x$ e $y$ e que $x$ e $y$ são as extremidades de $xy$.
%Se $xy$ é uma aresta, então dizemos que $x$ e $y$ são adjacentes.

%\missingfigure[figwidth=6cm]{Petersepessoasn Graph}

%A quantidade de vértices em um grafo $G$ será denotada por $v(G) = \card{V(G)}$ e comumnente será indicada por $n = v(G)$. A quantidade de arestas de $G$ é denotada por $e(G) = \card{E(G)}$ e comumente será referida por $m = e(G)$. Note que por definição temos $0 \leq e(G) \leq \binom{n}{2} = \frac{n(^2 - n)}{2}$.

%\begin{example}[Grafos Completos]
%Um grafo é dito completo quando possui todas as arestas possíveis. Denotamos o grafo completo com $n$ vértices por $K_n = (V,V^{(2)})$, onde $V$ é algum conjunto de vértices com $n = |V|$ elementos. O grafo $K_3$ é também chamado de triângulo.

%\missingfigure[figwidth=10cm]{$K_n$ para $n = 1,2,3,4,5,6$}
%\end{example}

%Se $v$ é um vértice de um grafo $G$, então a vizinhança $N_G(v)$ de $v$, o conjunto dos vértices adjacentes à $v$. Quando o grafo for conhecido pelo contexto, utilizaremos apenas $N(v)$. O grau de um vértice $d_G(v)$ é o tamanho de sua vizinhança, isto é, $d(v) = \card{N(v)}$.

%O complemento de um grafo $G$ é o grafo $\comp{G} = (V(G), V(G)^{(2)} \setminus E(G))$, isto é, o grafo com os mesmos vértices de $G$ e arestas complementares à $G$, assim, $\comp{\comp{G}} = G$.

%Seja $G$ um grafo, dizemos que $H$ é um subgrafo de $G$ se $H$ for um grafo tal que $V(H) \subset V(G)$ e $E(H) \subset E(G)$. Quando $H$ é subgrafo de $G$, às vezes dizemos que $G$ possui $H$, ou que existe uma cópia de $H$ em $G$.

%Finalmente, se $G$ é um grafo, então uma coloração de arestas de $G$ em $k$ cores é um mapa entre $E(G)$ e um conjunto de $k$ cores, isto é, uma função $c: E(G) \to \{ C_1, \dots, C_k\}$. Uma coloração de arestas em $k$ cores também é chamada de $k$-coloração de arestas. Note que se considerarmos apenas arestas de cor $C$ em uma coloração de arestas, temos um subgrafo de $G$, que será denotado por $G_C$.

%\missingfigure[figwidth=10cm]{Exemplo de grafo, uma 2-coloração de arestas }

%%%%%%%%%%%%%%%%%%%%%%%%%%%%%%%%%%%%%%%%%%%%%%%%%%%%%%%%%%%%%%%%%%%%%%%%%%%%%%%%

\section{Números de Ramsey}

No início do capítulo, vimos o problema das mesas, que nos diz que em mesas de seis alunos, três alunos se conhecem ou três alunos não se conhecem. Consideremos agora um grafo $G$ no qual os vértices são os seis alunos na mesa e uma aresta está presente entre dois alunos quando eles se conhecem. Note que $\comp{G}$ é o grafo em que dois alunos são adjacentes quando não se conhecem. O problema das mesas então nos diz que se $G$ possui 6 vértices então $G$ ou $\comp{G}$ possui um triângulo.

Observar o mesmo fato em termos de colorações é ainda mais vantajoso. Novamente considere $G$ como o grafo onde os vértices são as seis pessoas na festa, mas desta vez, considere o grafo completo. Realizamos uma coloração das arestas de $G$ em duas cores, $c: E(G) \to \{ R,B \}$ ($R$ e $B$ denotam vermelho e azul respectivamente), de forma que $c(xy) = R$ se $x$ e $y$ são pessoas que se conheçem e $c(xy) = B$ se $x$ e $y$ são pessoas que não se conhecem. Assim queremos ver que $G$ possui um triângulo monocromático, isto é, $G_R$ ou $G_B$ possuem um triangulo.

\begin{theorem}
Se $G$ é um grafo com 6 vértices e $c: E(G) \to \{ R,B\}$ é uma coloração de arestas, então $G$ possui um triângulo monocromático.
\end{theorem}
\begin{proof}
Seja $v \in V(G)$ um vértice qualquer e considere a sua vizinhança $N(v)$. O vértice $v$ se liga aos vértices de $N(v)$ utilizando arestas vermelhas ou azuis, logo $N(v) = N_R(v) \union N_B(v)$. Como o grau de $v$ é 5, então $d_R(v) + d_B(v) = 5$. Pelo princípio das casas dos pombos, alguma cor dentre $R$ e $B$, digamos $R$, é tal que $d_R(v) >= 3$. Agora observe que se dois vértices de $N_R(v)$ se conectam por uma aresta vermelha, unidas ao vértice $v$ eles formam um triângulo vermelho. Se isto não ocorrer, toda aresta interna à $N_R(v)$ é azul, o que garante que existe um triângulo azul em $v$. De qualquer caso, $G$ possui um triângulo monocromático.

\missingfigure[figwidth=10cm]{Figura com v destacado, uma vizinhança de 3 vértices e um triangulo monocromático }
\end{proof}

Esta formulação é interessante porque não nos permitiu provar o problema da festa, mas nos permite estudar melhor generalizações do mesmo problema.

%%%%%%%%%%%%%%%%%%%%%%%%%%%%%%%%%%%%%%%%%%%%%%%%%%%%%%%%%%%%%%%%%%%%%%%%%%%%%%%%

%\section{Mais de Grafos}

%\todo[inline]{Isomorfismos}

%\todo[inline]{Subgrafo}
%\todo[inline]{Subgrafo Induzido}

%\begin{example}[Caminhos]
%Um grafo $G$ é um caminho de comprimento $n$ se podemos ordenar os vértices $V(G) = \{ v_0, \dots, v_n\}$ de forma que as arestas de $G$ são exatamente $v_iv_{i+1}$ para $i = 0,\dots, n$. Os vértices $v_0$ e $v_n$ são os extremos do caminho e tem grau 1, os demais possuem grau 2. Denotamos um caminho de comprimento $n$ por $P_n$. Note que $v(P_n) = n+1$ e $e(P_n) = n$.
%\missingfigure[figwidth=6cm]{Alguns $P_n$}
%\end{example}

%\todo[inline]{Ciclos}


%\todo[inline]{Conexidade}
%\todo[inline]{Árvores}

%\todo[inline]{Cópia Disjunta}



%%%%%%%%%%%%%%%%%%%%%%%%%%%%%%%%%%%%%%%%%%%%%%%%%%%%%%%%%%%%%%%%%%%%%%%%%%%%%%%%
