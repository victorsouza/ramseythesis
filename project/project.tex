%%%%%%%%%%%%%%%%%%%%%%%%%%%%%%%%%%%%%%%%%%%%%%%%%%%%%%%%%%%%%%%%%%%%%%%%%%%%%%%%
%% Projeto Final de Graduação
%% Aluno: Victor Seixas Souza
%% Orientadora: Christiane Neme Campos
%% Tema: Teoria de Ramsey em Grafos
%%%%%%%%%%%%%%%%%%%%%%%%%%%%%%%%%%%%%%%%%%%%%%%%%%%%%%%%%%%%%%%%%%%%%%%%%%%%%%%%
% arara: latex
% arara: bibtex
% arara: latex
% arara: latex
% arara: pdflatex
% arara: clean: {files: [project.aux, project.log, project.blg, project.bbl]}
% arara: clean: {files: [project.toc, project.dvi]}
%%%%%%%%%%%%%%%%%%%%%%%%%%%%%%%%%%%%%%%%%%%%%%%%%%%%%%%%%%%%%%%%%%%%%%%%%%%%%%%%
% DOCUMENT CLASS
\documentclass[a4paper,10pt]{article}

%%%%%%%%%%%%%%%%%%%%%%%%%%%%%%%%%%%%%%%%%%%%%%%%%%%%%%%%%%%%%%%%%%%%%%%%%%%%%%%%
% PACKAGES

\usepackage[brazilian]{babel}
\usepackage[utf8]{inputenc}
\usepackage[T1]{fontenc}
\usepackage{amsmath}
\usepackage{amssymb}
\usepackage{amsfonts}
\usepackage{amsthm}
\usepackage{fullpage}

%%%%%%%%%%%%%%%%%%%%%%%%%%%%%%%%%%%%%%%%%%%%%%%%%%%%%%%%%%%%%%%%%%%%%%%%%%%%%%%%
% CONFIGURATION

\pagestyle{plain}
\bibliographystyle{plain}

%%%%%%%%%%%%%%%%%%%%%%%%%%%%%%%%%%%%%%%%%%%%%%%%%%%%%%%%%%%%%%%%%%%%%%%%%%%%%%%%
% FILE INFO

\author{Aluno: Victor Seixas Souza\\
Supervisora: Christiane Neme Campos
}
\date{\today}
\title{Projeto Final de Graduação\\
\Huge{Introdução à Teoria de Ramsey em Grafos}}

%%%%%%%%%%%%%%%%%%%%%%%%%%%%%%%%%%%%%%%%%%%%%%%%%%%%%%%%%%%%%%%%%%%%%%%%%%%%%%%%

\begin{document}

%%%%%%%%%%%%%%%%%%%%%%%%%%%%%%%%%%%%%%%%%%%%%%%%%%%%%%%%%%%%%%%%%%%%%%%%%%%%%%%%
% TITLE

\maketitle

%%%%%%%%%%%%%%%%%%%%%%%%%%%%%%%%%%%%%%%%%%%%%%%%%%%%%%%%%%%%%%%%%%%%%%%%%%%%%%%%
% MAIN CONTENT

\begin{abstract}
Este documento apresenta um projeto para a disciplina MC030 - Projeto Final de Graduação, a ser desenvolvido no segundo semestre de 2016, pelo aluno Victor Seixas Souza, sob a orientação da Prof.a Christiane Neme Campos.
\end{abstract}

%%%%%%%%%%%%%%%%%%%%%%%%%%%%%%%%%%%%%%%%%%%%%%%%%%%%%%%%%%%%%%%%%%%%%%%%%%%%%%%%

\section{Introdução e Objetivos}

A Teoria de Ramsey é uma área da matemática que unifica o tema: desordem completa é impossível. Mais especificamente, observamos que se uma estrutura é grande o suficiente, então ela possui uma subestrutura bem especial e ordenada. Este fenômeno ocorre em diversos campos da matemática, como Combinatória, Geometria e Teoria dos Números. Este projeto aborda conceitos básicos e alguns dos resultados clássicos em Teoria de Ramsey aplicada a grafos.

O exemplo mais simples de tal corpo teórico é frequentemente apresentado na seguinte história: em uma festa com pelo menos seis pessoas, três delas se conhecem mutuamente ou três delas não se conhecem mutuamente. Se enxergarmos a relação de conhecer como simétrica, o mesmo pode ser traduzido para linguagem de Teoria de Grafos como: todo grafo com pelo menos seis vértices possui um triângulo, ou, então, o seu grafo complementar possui um triângulo. A Teoria de Ramsey inicia-se pela generalização sucessiva deste enunciado para grafos e hipergrafos.

A Teoria de Ramsey tem seu nome em homenagem ao matemático e filósofo britânico Frank P. Ramsey, por seu trabalho, em lógica, publicado em 1930 \cite{ramsey}, mas apenas adquiriu um corpo teórico coeso na década de 1970. A área vem recebendo grande atenção nos últimos vinte anos por suas conexões com diversos campos da matemática e, ainda assim, muito dos seus problemas fundamentais permanecem sem solução. Além disso, não muito da teoria propagou-se para os livros didáticos em nível de graduação. Entretanto, é possível abordar parte da teoria sem recorrer ao ferramental mais avançado.

Considerando a lacuna da literatura citada anteriormente, este projeto tem por objetivo a elaboração de um texto introdutório à Teoria de Ramsey em grafos, em língua portuguesa. Além disso, planeja-se completar este texto com a apresentação de um ou dois tópicos mais avançados da área, que serão selecionados dentre: o método probabilístico; o lema da regularidade de Szemerédi; ou aplicações em Teoria dos Números.

%%%%%%%%%%%%%%%%%%%%%%%%%%%%%%%%%%%%%%%%%%%%%%%%%%%%%%%%%%%%%%%%%%%%%%%%%%%%%%%%

\section{Metodologia e Plano de Trabalho}

Os métodos que serão aplicados são os tradicionais utilizados na pesquisa em Combinatória. Inicialmente, o estudo se dará por meio da leitura de livros-texto \cite{alon, bollobas, bondy, diestel, graham, graham_rudiments, nesetril}, que apresentam a teoria de forma mais paulatina. Além disso, esta fase permitirá um melhor entendimento do que existe, na literatura, sobre a teoria. Em um segundo momento, o estudo terá como focos os conceitos basícos e resultados clássicos. Posteriormente, serão escolhidos dois tópicos avançados e, estes, serão estudados e incluídos no texto final. A Tabela \ref{tab:plano} exibe o plano de  trabalho proposto.

\begin{table}[h!]
\centering
\caption{Plano de Trabalho}
\label{tab:plano}
\begin{tabular}{|l|l|}
\hline
Mês      & Atividade                                \\\hline\hline
Agosto   & Leitura preliminar da bibliografia       \\
Setembro & Conceitos básicos e resultados Clássicos \\
Outubro  & Tópico Avançado 1                        \\
Novembro & Tópico Avançado 2                        \\
Dezembro & Ajustes Finais                           \\\hline
\end{tabular}
\end{table}

Parte das atividades desta disciplina inclui a preparação de uma monografia sobre o trabalho desenvolvido. Por esta razão, foram planejadas entregas parciais deste documento conforme explicitado na Tabela \ref{tab:prazo}.

\begin{table}[h!]
\centering
\caption{Prazos para entregas}
\label{tab:prazo}
\begin{tabular}{|l|l|}
\hline
Data           & Entrega                    \\\hline\hline
26 de Agosto   & Esqueleto do Relatório     \\
30 de Setembro & Primeiro Relatório Parcial \\
28 de Outubro  & Segundo Relatório Parcial  \\
25 de Novembro & Terceiro Relatório Parcial \\
07 de Dezembro & Relatório Final            \\
21 de Dezembro & Relatório Final Corrigido  \\\hline
\end{tabular}
\end{table}

Ao longo de todo este período, ocorrerão reuniões de trabalho semanais com a supervisora, que visam ao acompanhamento e ao direcionamento do desenvolvimento do projeto.

%%%%%%%%%%%%%%%%%%%%%%%%%%%%%%%%%%%%%%%%%%%%%%%%%%%%%%%%%%%%%%%%%%%%%%%%%%%%%%%%

\section{Resultados Esperados}

O trabalho a ser desenvolvido tem por objetivo preencher os requisitos
exigidos pela disciplina MC030 -- Projeto Final de Graduação por meio
da elaboração de uma monografia sobre o tema escolhido. Além disso,
espera-se que ao longo deste semestre, o aluno se desenvolva do ponto
de vista científico e que isto se reflita na forma como ele consegue
exprimir e manipular os resultados a que vier a ser exposto.

%%%%%%%%%%%%%%%%%%%%%%%%%%%%%%%%%%%%%%%%%%%%%%%%%%%%%%%%%%%%%%%%%%%%%%%%%%%%%%%%
% REFERENCES

\bibliography{references}

%%%%%%%%%%%%%%%%%%%%%%%%%%%%%%%%%%%%%%%%%%%%%%%%%%%%%%%%%%%%%%%%%%%%%%%%%%%%%%%%

\end{document}

%%%%%%%%%%%%%%%%%%%%%%%%%%%%%%%%%%%%%%%%%%%%%%%%%%%%%%%%%%%%%%%%%%%%%%%%%%%%%%%%
